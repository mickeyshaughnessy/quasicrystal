
\documentclass[11pt, oneside]{article}   	% use "amsart" instead of "article" for AMSLaTeX format
\usepackage{geometry}                		% See geometry.pdf to learn the layout options. There are lots.
\geometry{letterpaper}                   		% ... or a4paper or a5paper or ... 
%\geometry{landscape}                		% Activate for rotated page geometry
%\usepackage[parfill]{parskip}    		% Activate to begin paragraphs with an empty line rather than an indent
\usepackage{graphicx}				% Use pdf, png, jpg, or eps§ with pdflatex; use eps in DVI mode
								% TeX will automatically convert eps --> pdf in pdflatex		
\usepackage{amssymb}

%SetFonts

%SetFonts


\title{Brief Article}
\author{The Author}
%\date{}							% Activate to display a given date or no date

\begin{document}
\maketitle

\begin{abstract}
We carry out scattering calculations on a one-dimensional point like arrangement of atoms, $\mathbf{\chi(x)}$, related to the distribution of prime numbers by a shift operation making the atomic density approximately constant. 
We show how the Riemann Zeta Function (RZF) naturally parameterizes the analytic structure of the scattering amplitude. 
Inspired by Dyson's suggestion \ref[Dyson] we present a simple proof that the non-trivial zeros of the RZF all lie along the line $Re = 1/2$ in the complex plane.
\end{abstract}

\section{Introduction}

\subsection{Fourier Transform}
The Fourier transform of $V(x)$ is $\hat{V}(k)$ 

\begin{equation}
\hat{V}(k) = \int_{-\infty}^{\infty}V(x)e^{-i2\pi kx}dx
\end{equation}

A one-dimensional scattering potential may be of the form 
\begin{equation}
V(x) = \sum_{x_n \in X}\delta_D(x - x_n)
\end{equation} 
 
where the $x_n$ are elements of a countable set of real numbers. 
Then $V(x)$ is called a tempered distribution.
 For certain $V(x)$ it is the case that it's Fourier transform, $\hat{V}(k)$, is also contain a tempered distribution. 
  
\begin{equation}
 \label{eq: RiemannFourier}
 \mathcal{F}\left \{V(x)\right \} = \mathcal{F}\left \{ \sum_{x_n \in X}\delta_D(x - x_n) \right \} = \hat{h}(k) +  \sum_{k_m \in X^{*}} V_{m} \delta_D(k - k_{m}),
\end{equation}

If $\hat{h}(k) = 0$ everywhere, then $V(x)$ is called a quasicrystal.
By applying the Fourier transform a second time to $\hat{V}(k)$, it is evident that $\hat{V}(k)$ is also a quasicrystal when $V(x)$ is a quasicrystal. 






In general, $\hat{V}(k) =\hat{g}(k) + \hat{h}(k)$ may have a discrete component, $\hat{g}(k)$  and a continuous component,$\hat{h}(k)$.

\begin{equation}
 \label{eq: RiemannFourier}
 \mathcal{F}\left \{ \sum_{x_n \in X}\delta_D(x - x_n) \right \} = h(k) +  \sum_{k_m \in X^{*}} F_{m} \delta_D(k - k_{m}),
\end{equation}.


\subsection{Wave scattering}
Consider a scattering potential, V(x), which is distribution of Dirac delta functions along the positive real line, one at each prime number. 
The delta functions are located at integers and so they have spacing at least 1.

\subsection{The Quasicrystal $\chi$ and Wave Scattering}
Following \ref{Varma2016} we define a specific quasicrystalline arrangement of atoms suitable for scattering calculations.
Consider a scattering potential, V(x), which is distribution of Dirac delta functions along the positive real line, one at each prime number. 
The delta functions are located at integers and so they have spacing at least 1.
It is well known that the prime counting function, $\pi(x)$, is approximately $\frac{x}{log(x)}$

\begin{figure}[htbp]
\begin{center}

\caption{default}
\label{default}
\end{center}
\end{figure}


V(k) is a function on the complex plane, $k = k_{Re} + ik_{Im}$
If V(k) is holomorphic, we can compute it by contour integration from V(x).


\section{Method}
\section{Results}
\section{Proof that all nontrivial zeros of the RZF lie on the line $Re = \frac{1}/{2}$}
\section{Conclusion}
\section{Appendix 1



\end{document}  

