
\documentclass[11pt, oneside]{article}   	% use "amsart" instead of "article" for AMSLaTeX format
\usepackage{geometry}                		% See geometry.pdf to learn the layout options. There are lots.
\usepackage{animate}
\geometry{letterpaper}                   		% ... or a4paper or a5paper or ... 
%\geometry{landscape}                		% Activate for rotated page geometry
%\usepackage[parfill]{parskip}    		% Activate to begin paragraphs with an empty line rather than an indent
\usepackage{graphicx}				% Use pdf, png, jpg, or eps§ with pdflatex; use eps in DVI mode
								% TeX will automatically convert eps --> pdf in pdflatex		
\usepackage{amssymb}

%SetFonts

%SetFonts


\title{Quasicrystal Scattering and the Riemann Zeta Function}
\author{Michael Shaughnessy}
%\date{}							% Activate to display a given date or no date

\begin{document}
\maketitle

\begin{abstract}
I carry out scattering calculations against a one-dimensional point like arrangement of atoms, $\mathbf{\chi(x)}$, related to the distribution of prime numbers by a shift operation making the atomic density approximately constant. 
I show how the Riemann Zeta Function (RZF) naturally parameterizes the analytic structure of the scattering amplitude. 
% Inspired by Dyson's quasicrystal suggestion \ref{Dyson} I present a cryptic speculation in the Appendix, vaguely indicating how the real component of every non-trivial zero of the storied Riemann Zeta Function might be determined.
\end{abstract}

\section{Introduction}

There have been many explanations of the curious relationship between the prime numbers and the values of the non-trivial zeros of the Riemann Zeta Function (RZF).

\ref{Riemann, Selberg, Dyson, Zhang}

Freeman Dyson \ref{Baez} speculated about a possible path to determining the relationship between the real and the imaginary components of the non-trivial zeros of the RZF, using the concept of a quasicrystal.

A quasiperiodic crystal, or quasicrystal, is a structure that is ordered in some way but not strictly periodic. A spiral pattern is an example of a quasicrystal. 

Quasicrystals were experimentally observed by Shechtman in 1985 \ref{Shechtman1985}. 

B. Riemann showed the prime numbers are partially ordered and are not periodic \ref{Riemann1859} in 1859 and H. Von Mangoldt \ref{VonMangoldt1895} proved the explicit formula in 1895.

The explicit formula of Guinand and Weil \ref{Weil} is a formula for the Fourier transform of the RZF zeros as a sum over prime powers, plus additional terms.  


%\subsection{Fourier Transform}
%\begin{figure}
 %   \centering
%    \animategraphics[width=0.8\linewidth,autoplay,loop]{627}{../animations/FT_animation-}{0}{626}
%    \caption{Fourier Transform animated. Open with Adobe Acrobat Reader to watch}
%    \label{fig:FT_animation}
%!TEX encoding = UTF-8 Unicode\end{figure}

The Fourier transform of $V(x)$ is $\hat{V}(k)$ 

\begin{equation}
\hat{V}(k) = \int_{-\infty}^{\infty}V(x)e^{-i2\pi kx}dx
\end{equation}

A one-dimensional scattering potential may be of the form 
\begin{equation}
V(x) = \sum_{x_n \in X}\delta_D(x - x_n)
\end{equation} 
 
where the $x_n$ are elements of a countable set of real numbers. 
Then $V(x)$ is called a tempered distribution.


 For certain $V(x)$ of the form above it is the case that it's Fourier transform, $\hat{V}(k)$, also contains a tempered distribution. 
  
\begin{equation}
 \label{eq: RiemannFourier}
 \mathcal{F}\left \{V(x)\right \} = \hat{V}(k) = \mathcal{F}\left \{ \sum_{x_n \in X}\delta_D(x - x_n) \right \} = \hat{h}(k) +  \sum_{k_m \in X^{*}} \hat{V_{m}} \delta_D(k - k_{m}),
\end{equation}

If $\hat{h}(k) = 0$ everywhere, then $V(x)$ is called a quasicrystal.

By applying the Fourier transform a second time to $\hat{V}(k)$, it is evident that $\hat{V}(k)$ is also a quasicrystal when $V(x)$ is a quasicrystal. When all the $x_n$ lie along a line, the $k_m$ must also lie along a line in the complex plane - applying the Fourier transform twice gives us back our original $V(x)$.


where $\pi(x)$ is the prime counting function.



%In general, $\hat{V}(k) =\hat{g}(k) + \hat{h}(k)$ may have a discrete component, $\hat{g}(k)$  and a continuous component, $\hat{h}(k)$.

%\begin{equation}
 %\label{eq: RiemannFourier}
 %\mathcal{F}\left \{ \sum_{x_n \in X}\delta_D(x - x_n) \right \} = h(k) +  \sum_{k_m \in X^{*}} F_{m} \delta_D(k - k_{m}),
%\end{equation}.


%\subsection{Wave scattering}
%Consider a scattering potential, V(x), which is distribution of Dirac delta functions along the positive real line, one at each prime number. 
%The delta functions are located at integers and so they have spacing at least 1.

\subsection{Wave scattering on $\chi$}
Inspired by Varma's approach, \ref{Varma2016} I define a specific 1-dimensional arrangement of atoms suitable for scattering calculations through a normalization or shift operation yielding an approximately constant atomic density, $\chi$.

Consider the specific scattering potential, $\chi$
\begin{equation}
\chi(x) = \sum_{p, \text{prime}}^{\infty} \delta(x - \frac{x_p}{\pi(x_p)}).
\end{equation}


Unlike Varma, I apply the shift transformation directly to the real space atomic positions, as opposed to the k-space positions of the zeros of the RZF.

Consider a scattering potential, V(x), which is distribution of Dirac delta functions along the positive real line, one at each prime number. 
The delta functions are located at integers and so they have spacing at least 1.

Do they also have a maximum spacing? They do not \ref{PrimeSpacing}.

The exact expression \cite{Riemann} for $\pi(x)$ when $x>1$ is:

\begin{equation}
\pi(x) = \pi_0(x) - \frac{1}{2} = R(x) - \sum_{\rho}R(x^{\rho}) - \frac{1}{2}
\end{equation}

where

\begin{equation}
R(x) = \sum_{n=1}^{\infty}\frac{\mu(n)}{n}li(x^{\frac{1}{n}})
\end{equation}

where $\mu(x)$ is the M\"obius function, $li(x)$ is the logarithmic integral function, and $\rho$ runs over all the zeros of the RZF.

If the trivial zeros are collected and the sum is taken only over the non-trivial zeros, then

\begin{equation}
\pi_0(x) \approx R(x) - \sum_{\rho}R(x^{\rho}) - \frac{1}{\log(x)} + \frac{1}{\pi}\arctan(\frac{1}{\log(x)})
\end{equation}
 


The quantity $\pi(x)/x$ has units of density - it presents the density of atoms around $x$ in the specific scattering potential defined by $V(x)$.

%I normalize the positions of the atoms in $V(x)$ with a shift operation, such that the density of atoms becomes approximately constant, yielding a shifted tempered distribution with finite spacing and approximately constant density - I call this potential $\chi(x)$ and the shift operator $p$: $p([x_n]) = [x_n * \frac{1}{\pi(x_n)}] \sim [\log(x_n)]$.

The physical process of scattering a wave from a potential can be represented by applying the Fourier transform to the potential. The result is a function onto the space of wave momentum (reciprocal, k-, or dual space), often called the spectrum or scattering amplitude of the wave against the potential. The scattering amplitude can be measured by plotting the number of times a reflected wave arrives back at the wave source as a function of the wave momentum, $k$.

I carry out numerical scattering calculations on finite-length approximates (parameterized by the total number of atoms, $L_{\chi}$) of the potential $\chi(x)$.

\section{Numerical Method}

 Code for the scattering calculation is available at 
 
 https://github.com/mickeyshaughnessy/quasicrystal/blob/main/scattering.py
 
It is well known that $\pi(x) \sim \frac{x}{log(x)}$ and this is used as the prime density function in the results that follow.
 
I evaluate the Fourier transform numerically for both $\chi(x)$ and $\hat{\chi}(k)$ and observe the approximate locations of peaks in both.
 
 \section{Results}
 Figure 1. shows the computed scattering amplitudes as a function of wave momentum (k) for two finite approximates, whose total length is parameterized by $L_{\chi}$. Vertical red dashes are drawn at the approximate locations of the imaginary component of the nontrivial RZF zeros. These seem to align in a 1-to-1 fashion with the computed peaks in the spectrum.
 
\begin{figure}[htbp]
\begin{center}
    \includegraphics[width=0.8\linewidth]{../images/zoomed_scattering.png}
   
\caption{Scattering amplitudes for finite $L_{\chi}$. Vertical red lines indicate the positions of the imaginary part of the non-trivial zeros of the RZF.}
\label{default}
\end{center}
\end{figure}
 
\section{Exact Calculation using the Residue Theorem}

I compute the Fourier transforms of $\chi(x)$ and $\hat{\chi}(k)$ using a contour integral, making clear how the zeta function enters through the prime density function and the residue theorem.

I discuss the continuous component of both spectra as the contribution of the "prime at infinity".

%\section{Numerical  
%V(k) is a function on the complex plane, $k = k_{Re} + ik_{Im}$
%If V(k) is holomorphic, we can compute it by contour integration from V(x).




\section{Conclusion}
\section{Acknowledgements}
I gratefully acknowledge helpful conversations with CY Fong, John Baez, Jamison Galloway, Robert Hayre, Chun Yen Lin, Charles Martin and Catalin Spataru.
\section{Appendix}



\end{document}  

