
\documentclass[11pt, oneside]{article}   	% use "amsart" instead of "article" for AMSLaTeX format
\usepackage{geometry}                		% See geometry.pdf to learn the layout options. There are lots.
\usepackage{animate}
\geometry{letterpaper}                   		% ... or a4paper or a5paper or ... 
%\geometry{landscape}                		% Activate for rotated page geometry
%\usepackage[parfill]{parskip}    		% Activate to begin paragraphs with an empty line rather than an indent
\usepackage{graphicx}				% Use pdf, png, jpg, or eps§ with pdflatex; use eps in DVI mode
								% TeX will automatically convert eps --> pdf in pdflatex		
\usepackage{amssymb}

%SetFonts

%SetFonts


\title{Quasicrystal Scattering and the Riemann Zeta Function}
\author{Michael Shaughnessy, CY Fong}
%\date{}							% Activate to display a given date or no date

\begin{document}
\maketitle

\begin{abstract}
We carry out scattering calculations against a one-dimensional point like arrangement of atoms, $\mathbf{\chi(x)}$, related to the distribution of prime numbers by a shift operation making the atomic density approximately constant. 
We show how the Riemann Zeta Function (RZF) naturally parameterizes the analytic structure of the scattering amplitude. 
Inspired by Dyson's suggestion \ref[Dyson] we present a simple proof that the non-trivial zeros of the RZF all lie along the line $Re = 1/2$ in the complex plane.
\end{abstract}

\section{Introduction}

\subsection{Fourier Transform}
\begin{figure}
    \centering
    \animategraphics[width=0.8\linewidth,autoplay,loop]{627}{../animations/FT_animation-}{0}{626}
    \caption{Fourier Transform animated. Open with Adobe Acrobat Reader to watch}
    \label{fig:FT_animation}
\end{figure}

The Fourier transform of $V(x)$ is $\hat{V}(k)$ 

\begin{equation}
\hat{V}(k) = \int_{-\infty}^{\infty}V(x)e^{-i2\pi kx}dx
\end{equation}

A one-dimensional scattering potential may be of the form 
\begin{equation}
V(x) = \sum_{x_n \in X}\delta_D(x - x_n)
\end{equation} 
 
where the $x_n$ are elements of a countable set of real numbers. 
Then $V(x)$ is called a tempered distribution.
 For certain $V(x)$ of the form above it is the case that it's Fourier transform, $\hat{V}(k)$, also contains a tempered distribution. 
  
\begin{equation}
 \label{eq: RiemannFourier}
 \mathcal{F}\left \{V(x)\right \} = \hat{V}(k) = \mathcal{F}\left \{ \sum_{x_n \in X}\delta_D(x - x_n) \right \} = \hat{h}(k) +  \sum_{k_m \in X^{*}} \hat{V_{m}} \delta_D(k - k_{m}),
\end{equation}

If $\hat{h}(k) = 0$ everywhere, then $V(x)$ is called a quasicrystal.
By applying the Fourier transform a second time to $\hat{V}(k)$, it is evident that $\hat{V}(k)$ is also a quasicrystal when $V(x)$ is a quasicrystal. When all the $x_n$ lie along a line, the $k_m$ must also lie along a line in the complex plane, because applying the Fourier transform twice gives us back our original $V(x)$.






%In general, $\hat{V}(k) =\hat{g}(k) + \hat{h}(k)$ may have a discrete component, $\hat{g}(k)$  and a continuous component, $\hat{h}(k)$.

%\begin{equation}
 %\label{eq: RiemannFourier}
 %\mathcal{F}\left \{ \sum_{x_n \in X}\delta_D(x - x_n) \right \} = h(k) +  \sum_{k_m \in X^{*}} F_{m} \delta_D(k - k_{m}),
%\end{equation}.


%\subsection{Wave scattering}
%Consider a scattering potential, V(x), which is distribution of Dirac delta functions along the positive real line, one at each prime number. 
%The delta functions are located at integers and so they have spacing at least 1.

\subsection{The Quasicrystal $\chi$ and Wave Scattering}
Inspired by Varma's approach, \ref{Varma2016} we define a specific quasicrystalline arrangement of atoms suitable for scattering calculations.
Unlike Varma, we apply the shift transformation directly to the real space atomic positions, as opposed to the k-space positions of the zeros of the RZF.

Consider a scattering potential, V(x), which is distribution of Dirac delta functions along the positive real line, one at each prime number. 
The delta functions are located at integers and so they have spacing at least 1.
Do they also have a maximum spacing? The answer is they do not \ref{PrimeSpacing}

The exact expression \cite{Riemann} for $\pi(x)$ when $x>1$ is:

\begin{equation}
\pi(x) = \pi_0(x) - \frac{1}{2} = R(x) - \sum_{\rho}R(x^{\rho}) - \frac{1}{2}
\end{equation}

where

\begin{equation}
R(x) = \sum_{n=1}^{\infty}\frac{\mu(n)}{n}li(x^{\frac{1}{n}})
\end{equation}

where $\mu(x)$ is the M\"obius function, $li(x)$ is the logarithmic integral function, and $\rho$ runs over all the zeros of the RZF.

If the trivial zeros are collected and the sum is taken only over the non-trivial zeros, then

\begin{equation}
\pi_0(x) \approx R(x) - \sum_{\rho}R(x^{\rho}) - \frac{1}{\log(x)} + \frac{1}{\pi}\arctan(\frac{1}{\log(x)}
\end{equation}
 
It is well known that $\pi(x)$, is also approximately $\frac{x}{log(x)}$ in a rougher approximation.

The quantity $\pi(x)/x$ has units of density - it presents the density of atoms around $x$ in the specific scattering potential defined by $V(x)$.

We normalize the positions of the atoms in $V(x)$ with a shift operation, such that the density of atoms becomes approximately constant, yielding a shifted tempered distribution with finite spacing and approximately constant density - we call this potential $\chi(x)$.




%V(k) is a function on the complex plane, $k = k_{Re} + ik_{Im}$
%If V(k) is holomorphic, we can compute it by contour integration from V(x).


\section{Method}
\section{Results}

\begin{figure}[htbp]
\begin{center}
    \includegraphics[width=0.8\linewidth]{../images/zoomed_scattering.png}
   
\caption{Scattering amplitude for finite $L_{\chi}$}
\label{default}
\end{center}
\end{figure}


\section{Proof that all nontrivial zeros of the RZF lie on the line $Re = \frac{1}/{2}$}
\section{Conclusion}
\section{Appendix 1}



\end{document}  

