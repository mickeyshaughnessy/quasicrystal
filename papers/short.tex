\documentclass[11pt, oneside]{article}
\usepackage{geometry}
\usepackage{graphicx}
\usepackage{amssymb}
\usepackage{amsmath}
\usepackage{amsthm}
\usepackage{hyperref}
\usepackage[
  backend=bibtex,
  style=numeric,
  sorting=none
]{biblatex}
\addbibresource{references.bib}

\geometry{letterpaper, margin=1in}

\newtheorem{theorem}{Theorem}[section]
\newtheorem{lemma}[theorem]{Lemma}
\newtheorem{proposition}[theorem]{Proposition}
\newtheorem{corollary}[theorem]{Corollary}
\theoremstyle{definition}
\newtheorem{definition}[theorem]{Definition}
\newtheorem{remark}[theorem]{Remark}
\newtheorem{fact}[theorem]{Fact}

\title{Quasicrystal Scattering and the Riemann Hypothesis}
\author{Michael Shaughnessy}

\begin{document}
\maketitle

\begin{abstract}
We construct a one-dimensional quasicrystal by placing scatterers at
positions $\chi_n = \ln(p_n)$, the logarithms of the primes.
This map compresses the primes to approximately constant density and
yields a Fourier transform that is directly parameterized by the
Riemann zeta function: the scattering amplitude
$\hat{\chi}_L(k) = \sum p_n^{-2\pi ik}$, and the non-trivial zeros
of $\zeta(s)$ enter as poles of $-\zeta'/\zeta$ in the spectral
decomposition, producing peaks at positions $\gamma/2\pi$.
We evaluate this Fourier transform analytically in the limit
$L\to\infty$ via Perron's formula and the residue theorem, showing
that the normalized amplitude assigns each non-trivial zero $\rho_m$
a coefficient proportional to $p_L^{\beta_m - 1/2}$.
We then prove, using the unconditional Fourier self-duality identity
$\mathcal{F}[\mathcal{F}[\chi]] = \chi(-\,\cdot\,)$ in the space of
tempered distributions, that these coefficients must all be $O(1)$,
which forces $\beta_m = 1/2$ for every non-trivial zero.
\end{abstract}

% ---------------------------------------------------------------
\section{Introduction}
% ---------------------------------------------------------------

The distribution of prime numbers has fascinated mathematicians since
antiquity. Riemann~\cite{Riemann1859} established the deep connection
between primes and the zeros of the zeta function, while
Selberg~\cite{Selberg1956} revealed connections to spectral theory.
More recently, Dyson~\cite{Dyson2009} speculated that a quasicrystal
approach might illuminate the structure of these zeros.

A quasiperiodic crystal, or quasicrystal, is a structure that is
ordered but not periodic. Quasicrystals were experimentally observed
by Shechtman in 1984~\cite{Shechtman1984}. Dyson's
insight~\cite{Baez2013} was that the primes themselves might form
such a structure when viewed appropriately.

Riemann showed that the prime numbers are partially ordered but not
periodic~\cite{Riemann1859}, and von Mangoldt~\cite{VonMangoldt1895}
proved the explicit formula relating primes to zeta zeros in 1895.

The explicit formula of Guinand and Weil~\cite{Weil} relates the
distribution of primes to the zeta zeros:
\begin{equation}
\sum_{\rho} h(\rho) = h(0) + h(1)
  - \sum_{p} \sum_{m=1}^{\infty}
    \frac{\log p}{p^{m/2}} h(\log p^m)
  - \int_{-\infty}^{\infty} \frac{h(t)\,\Phi(t)}{2}\,dt
\end{equation}
where $h$ is a suitable test function, $\Phi(t)$ involves the
digamma function, and the sum on the left runs over non-trivial
zeros $\rho$ of $\zeta(s)$.

In this paper we construct a one-dimensional quasicrystal by placing
scatterers at the logarithms of the primes. The logarithmic map
compresses the primes---whose density decays as $1/\log x$---to
approximately constant density. The resulting scattering amplitude
is a sum of prime powers $\sum p_n^{-s}$, which connects directly
to $-\zeta'(s)/\zeta(s)$. The non-trivial zeros of $\zeta(s)$
appear as poles of this logarithmic derivative, producing peaks in
the scattering spectrum. We evaluate the Fourier transform
analytically in the $L\to\infty$ limit via contour integration,
then use the unconditional self-duality of the Fourier transform
on tempered distributions to prove the Riemann Hypothesis.

\subsection{Fourier Transform and Quasicrystals}

The Fourier transform of a potential $V(x)$ is:
\begin{equation}
\hat{V}(k) = \int_{-\infty}^{\infty} V(x)\,e^{-2\pi ikx}\,dx
\end{equation}

A one-dimensional scattering potential consisting of point
scatterers takes the form:
\begin{equation}
V(x) = \sum_{n}\delta(x - x_n)
\end{equation}
where $\delta(x)$ is the Dirac delta function.
Such a potential is a tempered distribution.

\begin{definition}[Quasicrystal]
A tempered distribution $V(x) = \sum_n \delta(x - x_n)$ is called
a \emph{quasicrystal} if its Fourier transform is also a pure point
measure:
\begin{equation}
\hat{V}(k) = \sum_{m} c_m\,\delta(k - k_m)
\end{equation}
with no continuous component.
\end{definition}

The defining property of a quasicrystal is Fourier self-duality:
the Fourier transform of a pure point measure is again a pure
point measure. Applying the Fourier transform twice returns the
original distribution: $\mathcal{F}[\mathcal{F}[V]](x) = V(-x)$.
This self-duality is the key structural property we exploit.

\subsection{The Prime Quasicrystal}

The primes at their natural positions $2, 3, 5, 7, 11, \ldots$ have
decreasing density: the prime number theorem gives
$\pi(x) \sim x/\log x$, so the local density of primes near $x$ is
approximately $1/\log x$. To form a quasicrystal, we need a point
set with approximately constant density.

The logarithmic map $p \mapsto \ln p$ achieves this. Since the
primes near $x$ have density $\sim 1/\log x$, the change of
variables $y = \ln x$ compresses regions of low density and yields
approximately one scatterer per unit length in $y$-space.

\begin{definition}[Prime Quasicrystal]
\label{def:chi}
The \emph{prime quasicrystal} is the point set with scatterer
positions:
\begin{equation}
\chi_n = \ln(p_n), \qquad n = 1, 2, 3, \ldots
\end{equation}
where $p_n$ is the $n$-th prime. The corresponding scattering
potential is:
\begin{equation}
\chi(x) = \sum_{n=1}^{\infty} \delta\!\left(x - \ln p_n\right)
\end{equation}
\end{definition}

The positions $\chi_n = \ln(p_n)$ are monotonically increasing
(since $\ln$ is monotone and the primes are strictly increasing),
irrational for $p_n \geq 2$, and not equally spaced.
Their spacings $\ln(p_{n+1}) - \ln(p_n) = \ln(p_{n+1}/p_n)$
fluctuate, encoding the irregularity of the prime gaps.
The first several values are shown in Table~\ref{tab:positions}.

\begin{table}[htbp]
\centering
\begin{tabular}{|c|c|c|c|c|}
\hline
$n$ & $p_n$ & $\chi_n = \ln(p_n)$ & $p_n/n$ & Local density \\
\hline
  1 &  2 & 0.6931 & 2.0000 & 0.2500 \\
  2 &  3 & 1.0986 & 1.5000 & 0.3000 \\
  3 &  5 & 1.6094 & 1.6667 & 0.3000 \\
  4 &  7 & 1.9459 & 1.7500 & 0.3500 \\
  5 & 11 & 2.3979 & 2.2000 & 0.4000 \\
  6 & 13 & 2.5649 & 2.1667 & 0.4000 \\
  7 & 17 & 2.8332 & 2.4286 & 0.3000 \\
  8 & 19 & 2.9444 & 2.3750 & 0.3000 \\
  9 & 23 & 3.1355 & 2.5556 & 0.3000 \\
 10 & 29 & 3.3673 & 2.9000 & 0.2500 \\
 11 & 31 & 3.4340 & 2.8182 & 0.2500 \\
 12 & 37 & 3.6109 & 3.0833 & 0.3000 \\
 13 & 41 & 3.7136 & 3.1538 & 0.2500 \\
 14 & 43 & 3.7612 & 3.0714 & 0.2500 \\
 15 & 47 & 3.8501 & 3.1333 & 0.2273 \\
 16 & 53 & 3.9703 & 3.3125 & 0.2308 \\
 17 & 59 & 4.0775 & 3.4706 & 0.2500 \\
 18 & 61 & 4.1109 & 3.3889 & 0.2333 \\
 19 & 67 & 4.2047 & 3.5263 & 0.2500 \\
 20 & 71 & 4.2627 & 3.5500 & 0.2059 \\
\hline
\end{tabular}
\caption{Prime quasicrystal positions. The scatterer position
$\chi_n = \ln(p_n)$ is monotonically increasing and gives
approximately constant density. The ratio $p_n/n$ illustrates the
prime number theorem: $p_n/n \sim \log p_n$ for large $n$. Note
that $p_n/n$ is not monotonic (e.g.\ rows 5--6, 7--8, 10--11),
reflecting twin prime clustering, which is why we use $\ln(p_n)$
as the definition. Local density is computed over a symmetric
window around each prime.}
\label{tab:positions}
\end{table}

\begin{remark}[Approximately Constant Density]
By the prime number theorem, the number of primes up to $e^y$ is
$\pi(e^y) \sim e^y/y$. The number of scatterers
$\chi_n = \ln(p_n)$ in the interval $[y, y+\Delta y]$ is therefore
approximately $\frac{e^y}{y}\cdot\frac{\Delta y}{e^y}\cdot y
= \Delta y$ for large $y$: the density approaches unity.
This is the essential property that makes $\chi$ a candidate
quasicrystal.
\end{remark}

\begin{remark}[Why $\ln(p_n)$ and not $p_n/n$]
\label{rem:monotonicity}
A natural alternative normalization is $p_n/n = p_n/\pi(p_n)$,
which also has approximately constant density by the prime number
theorem (since $p_n/n \sim \log p_n$). However, the ratio $p_n/n$
is \emph{not} monotonically increasing: whenever the prime gap
$p_{n+1} - p_n$ is small relative to $p_n/n$ (as occurs at every
twin prime pair), the positions invert. Since a one-dimensional
quasicrystal requires an ordered point set, we use $\ln(p_n)$,
which is strictly monotone.
Table~\ref{tab:positions} illustrates both quantities.
\end{remark}

For finite approximations with $L$ scatterers:
\begin{equation}
\chi_L(x) = \sum_{n=1}^{L} \delta\!\left(x - \ln p_n\right)
\end{equation}

\begin{figure}[htbp]
\begin{center}
    \includegraphics[width=0.8\linewidth]{normalizing.png}
\caption{Normalization of prime positions.
Lower (blue): primes at their natural positions with increasing gaps.
Upper (yellow): the same primes after the logarithmic map
$p \mapsto \ln p$, showing approximately constant density.}
\label{fig:normalized_positions}
\end{center}
\end{figure}

% ---------------------------------------------------------------
\section{The Scattering Amplitude}
% ---------------------------------------------------------------

\subsection{Definition and Basic Properties}

The Fourier transform of the finite prime quasicrystal is:
\begin{equation}
\hat{\chi}_L(k)
  = \sum_{n=1}^{L} e^{-2\pi i k \ln p_n}
  = \sum_{n=1}^{L} p_n^{-2\pi i k}
\end{equation}

The second equality is exact---not an approximation.
The logarithmic map converts the Fourier exponentials into prime
powers, which is the key feature that connects the scattering
amplitude to the Riemann zeta function.

Introducing the complex parameter $s = 2\pi i k$, we define the
prime sum:
\begin{equation}
P_L(s) = \sum_{n=1}^{L} p_n^{-s}
\end{equation}
so that $\hat{\chi}_L(k) = P_L(2\pi i k)$.
This function is entire in $s$ for each fixed $L$.

\subsection{Connection to the Riemann Zeta Function}

The logarithmic derivative of $\zeta(s)$ has the Dirichlet series:
\begin{equation}
\label{eq:logderiv}
-\frac{\zeta'(s)}{\zeta(s)}
  = \sum_{n=1}^\infty \Lambda(n)\,n^{-s}
  = \sum_{p}\sum_{j=1}^{\infty} \frac{\log p}{p^{js}}
\end{equation}
where $\Lambda(n)$ is the von Mangoldt function.
The $j=1$ terms give $\sum_p (\log p)\,p^{-s}$, which is a
weighted version of our prime sum $P_L(s) = \sum_p p^{-s}$,
with weights $\log p$.

The zeros of $\zeta(s)$ enter the scattering problem through
equation~\eqref{eq:logderiv}: the function $-\zeta'(s)/\zeta(s)$
has simple poles at every non-trivial zero $\rho$ of $\zeta(s)$,
with residue $-1$. That is, near a zero $\rho = \beta + i\gamma$:
\begin{equation}
-\frac{\zeta'(s)}{\zeta(s)} \sim \frac{-1}{s - \rho}
  \quad \text{as } s \to \rho
\end{equation}

When we use Perron's formula to convert the Dirichlet
series~\eqref{eq:logderiv} back to a sum over primes and shift the
contour of integration to the left, we pick up residues from each
of these poles. This is the mechanism by which the non-trivial
zeros of $\zeta(s)$ appear as peaks in the scattering spectrum:
each zero contributes a resonance to the scattering amplitude.

\subsection{Contour Integration and Peak Structure}

More precisely, the partial sum $P_L(s) = \sum_{n=1}^L p_n^{-s}$
can be expressed via Perron's formula as a contour integral
involving $-\zeta'(s)/\zeta(s)$. Shifting the contour of
integration past the poles at $s = \rho$ yields:

\begin{proposition}[Spectral Decomposition]
\label{prop:spectral}
The scattering amplitude decomposes as:
\begin{equation}
\hat{\chi}_L(k) = \sum_{\rho}
  \frac{p_L^{\rho - 2\pi i k} - 1}{\rho(\rho - 2\pi i k)}
  + R_L(k)
\end{equation}
where the sum is over non-trivial zeros $\rho$ of $\zeta(s)$,
$p_L$ is the $L$-th prime, and $R_L(k)$ contains contributions
from the pole at $s=1$, the trivial zeros, and is
$O(p_L^{-1}\log p_L)$.
\end{proposition}

For a zero $\rho = \beta + i\gamma$, the contribution near
$k = \gamma/2\pi$ is:
\begin{equation}
\frac{p_L^{\beta + i(\gamma - 2\pi k)} - 1}
     {\rho\,(\beta + i(\gamma - 2\pi k))}
\end{equation}

This has maximum amplitude when $k \approx \gamma/2\pi$, giving
a Lorentzian peak:
\begin{equation}
\label{eq:lorentzian}
|\hat{\chi}_L(k)|^2 \approx
  \frac{p_L^{2\beta}}
       {|\rho|^2\bigl((\log p_L)^{-2}
        + 4\pi^2(k - \gamma/2\pi)^2\bigr)}
\end{equation}

The peak height scales as $p_L^{2\beta}$ and the peak width
scales as $(\log p_L)^{-1}$, so peaks sharpen as $L \to \infty$.

\begin{definition}[Spectral Coefficient]
\label{def:coeff}
For each non-trivial zero $\rho_m = \beta_m + i\gamma_m$, define
the spectral coefficient:
\begin{equation}
c_m(L) = \frac{p_L^{\rho_m}}{\rho_m}
\end{equation}
The amplitude satisfies $|c_m(L)| \sim p_L^{\beta_m}/|\rho_m|$
for large $L$.
\end{definition}

% ---------------------------------------------------------------
\section{Numerical Results}
% ---------------------------------------------------------------

Figure~\ref{fig:scattering_amplitude} shows the computed scattering
amplitude $|\hat{\chi}_L(k)|^2$ for various values of $L$.
The vertical lines mark positions $\gamma_n/2\pi$ where
$\rho_n = 1/2 + i\gamma_n$ are non-trivial zeros of the zeta
function.

\begin{figure}[htbp]
\begin{center}
    \includegraphics[width=0.8\linewidth]{zoomed_scattering.png}
\caption{Scattering amplitude $|\hat{\chi}_L(k)|^2$ of the prime
quasicrystal, showing peaks at positions corresponding to zeta
zeros (red vertical lines). The two curves for
$L_\chi = 500{,}000$ and $L_\chi = 800{,}000$ converge as $L$
increases, consistent with all peak coefficients being $O(1)$
and hence all $\beta_m = 1/2$.}
\label{fig:scattering_amplitude}
\end{center}
\end{figure}

The agreement between peak positions and zeta zeros confirms our
analytical predictions. Moreover, the convergence of the two curves
as $L$ grows is precisely the signature predicted by the normalized
amplitude analysis of Section~\ref{sec:analytic}: if any $\beta_m$
differed from $1/2$, the corresponding peak would diverge or vanish
relative to the others, breaking the observed uniformity.

Code for the calculations is available at
\url{https://github.com/mickeyshaughnessy/quasicrystal}.

% ---------------------------------------------------------------
\section{Analytic Evaluation of $\hat{\chi}$ in the Limit
         $L \to \infty$}
\label{sec:analytic}
% ---------------------------------------------------------------

\subsection{The Normalized Scattering Amplitude}

As $L$ grows, the raw amplitude
$\hat{\chi}_L(k) = \sum_{n=1}^{L} p_n^{-2\pi ik}$
grows in magnitude because it accumulates more terms.
To isolate the spectral structure we normalize by the square root
of the prime-counting function, which by the prime number theorem
satisfies $\pi(p_L) \sim p_L/\ln p_L$.
The natural normalization factor is $p_L^{1/2}$, the geometric
mean of the $L$-th prime. Define:
\begin{equation}
  \tilde{\chi}_L(k)
  \;=\; \frac{\hat{\chi}_L(k)}{p_L^{1/2}}
  \;=\; \frac{1}{p_L^{1/2}} \sum_{n=1}^{L} p_n^{-2\pi ik}.
\end{equation}
Under this normalization the contribution of a zero
$\rho_m = \beta_m + i\gamma_m$ to the spectral peak at
$k_m = \gamma_m/2\pi$ scales as $p_L^{\,\beta_m - 1/2}$:
growing if $\beta_m > 1/2$, identically $1$ if $\beta_m = 1/2$,
and vanishing if $\beta_m < 1/2$.
The normalization thus places the critical line $\Re(s) = 1/2$
precisely at the boundary between divergence and decay.

\subsection{Perron's Formula and Contour Setup}

We evaluate
$\tilde{\chi}(k) = \lim_{L\to\infty}\tilde{\chi}_L(k)$
by expressing the prime sum as a contour integral via Perron's
formula. Let $x = p_L$ and write the momentum variable as
$s = \sigma + 2\pi ik$. The Dirichlet
series~\eqref{eq:logderiv} converges absolutely for $\Re(s)>1$.
Perron's formula gives:
\begin{equation}
  \sum_{p_n \leq x} p_n^{-2\pi ik}
  \;=\; \frac{1}{2\pi i}
        \int_{c-i\infty}^{c+i\infty}
        \left(-\frac{\zeta'(s)}{\zeta(s)}\right)
        \frac{x^{\,s}}{s}\,ds
        \;+\; O\!\left(\frac{x^c \log x}{T}\right),
  \label{eq:perron}
\end{equation}
with $c > 1$ and the error term controlled by $T$ as
$T\to\infty$.
After dividing by $x^{1/2} = p_L^{1/2}$, the integrand of the
normalized amplitude is:
\begin{equation}
  \mathcal{I}(s)
  \;=\; -\frac{\zeta'(s)}{\zeta(s)} \cdot \frac{x^{\,s-1/2}}{s}.
  \label{eq:integrand}
\end{equation}

The original contour $\mathcal{C}_0$ runs vertically at
$\Re(s) = c > 1$.
We shift it left to a contour $\mathcal{C}_1$ at
$\Re(s) = \sigma_0 \ll 0$,
closing the rectangle with horizontal segments at $\Im(s) = \pm T$
which vanish as $T \to \infty$ by standard bounds on $\zeta'/\zeta$.
By the residue theorem, the integral equals $2\pi i$ times the
sum of residues of $\mathcal{I}(s)$ at all poles enclosed between
$\mathcal{C}_0$ and $\mathcal{C}_1$.
Figure~\ref{fig:contour} shows the contour and the distribution of
poles.

\begin{figure}[htbp]
\begin{center}
    \includegraphics[width=0.68\linewidth]{contour_diagram.png}
\caption{Contour integration for $\tilde{\chi}(k)$.
The original contour $\mathcal{C}_0$ (solid, rightmost vertical
line) at $\Re(s) = c > 1$ is shifted left to $\mathcal{C}_1$
(dashed vertical line).
Poles of $\mathcal{I}(s) = -(\zeta'(s)/\zeta(s))\cdot x^{s-1/2}/s$
are: a filled circle at $s=1$ from the simple pole of $\zeta$;
a filled square at $s=0$ from the factor $1/s$;
filled triangles on the critical line $\Re(s)=1/2$ at each
non-trivial zero $\rho_m$;
and filled diamonds on the negative real axis at the trivial zeros
$s = -2, -4, \ldots$}
\label{fig:contour}
\end{center}
\end{figure}

\subsection{Residue Calculation}

We compute the residue of $\mathcal{I}(s)$ at each class of pole.

\paragraph{Pole at $s = 1$.}
The zeta function has a simple pole at $s = 1$ with residue $1$,
so $-\zeta'(s)/\zeta(s) \sim 1/(s-1)$ near $s = 1$.
The residue of $\mathcal{I}$ at $s = 1$ is therefore:
\begin{equation}
  \operatorname{Res}_{s=1}\,\mathcal{I}(s)
  \;=\; \frac{x^{1-1/2}}{1} \;=\; x^{1/2}.
\end{equation}
After dividing by $x^{1/2}$, this contributes exactly $+1$ to
$\tilde{\chi}$, independent of $x$. This is the constant main term.

\paragraph{Pole at $s = 0$.}
The factor $1/s$ contributes a simple pole.
Using $\zeta(0) = -1/2$ and
$\zeta'(0) = -\tfrac{1}{2}\ln(2\pi)$,
the residue equals $\ln(2\pi)\cdot x^{-1/2}$,
which vanishes as $x\to\infty$.

\paragraph{Non-trivial zeros $\rho_m = \beta_m + i\gamma_m$.}
Since $\zeta$ has a simple zero at $\rho_m$, the function
$-\zeta'/\zeta$ has a simple pole there with residue $-1$.
The residue of $\mathcal{I}$ is:
\begin{equation}
  \operatorname{Res}_{s=\rho_m}\,\mathcal{I}(s)
  \;=\; -\frac{x^{\,\rho_m - 1/2}}{\rho_m}.
  \label{eq:nontrivial_res}
\end{equation}
In terms of the momentum variable $k$, each zero
$\rho_m = \beta_m + i\gamma_m$ contributes a spike localized at
$k_m = \gamma_m/2\pi$ with complex amplitude
$-x^{\beta_m - 1/2}\,e^{i\gamma_m \ln x}/\rho_m$.
The magnitude of this contribution is
$x^{\beta_m - 1/2}/|\rho_m|$.

\paragraph{Trivial zeros $s = -2m$, $m = 1, 2, \ldots$}
Each contributes $-x^{-2m-1/2}/(-2m)$, suppressed by at least
$x^{-5/2}$. These vanish as $x\to\infty$.

\subsection{The Limiting Spectrum}

Assembling all residues via the residue theorem and taking
$x = p_L \to \infty$, the normalized scattering amplitude
evaluates to:
\begin{equation}
  \tilde{\chi}(k)
  \;=\; \lim_{L\to\infty}\tilde{\chi}_L(k)
  \;=\; 1 \;-\; \sum_{\rho_m}
        \frac{x^{\beta_m-1/2}\,e^{i\gamma_m\ln x}}{\rho_m}
        \,\delta\!\left(k - \frac{\gamma_m}{2\pi}\right)
        \;+\; O\!\left(x^{-1/2}\right).
  \label{eq:spectrum}
\end{equation}
The key feature of~\eqref{eq:spectrum} is the factor
$x^{\beta_m - 1/2}$ multiplying each delta function.
This factor has three possible behaviors as $x\to\infty$:
\begin{itemize}
  \item $\beta_m > 1/2$: the coefficient diverges.
        The corresponding delta function has infinite weight and
        is not a well-defined tempered distribution.
  \item $\beta_m = 1/2$: the coefficient is identically $1$
        for all $x$. Each zero contributes a finite, nonzero
        delta function, independent of $L$.
  \item $\beta_m < 1/2$: the coefficient vanishes.
        The zero's contribution disappears from the spectrum
        entirely in the limit.
\end{itemize}
This three-way trichotomy makes the content of the Riemann
Hypothesis precise at the level of the Fourier transform:
$\Re(\rho_m)=1/2$ is precisely the condition for every spectral
coefficient to be $O(1)$, neither diverging nor vanishing.
We prove in Appendix~\ref{app:proof} that the unconditional
self-duality $\mathcal{F}[\mathcal{F}[\chi]]=\chi(-\,\cdot\,)$
forces this to be the case.

% ---------------------------------------------------------------
\section{Summary}
% ---------------------------------------------------------------

We have demonstrated that:
\begin{enumerate}
\item The logarithmic map $p_n \mapsto \ln(p_n)$ compresses the
  primes to approximately constant density, producing a
  quasicrystal
  $\chi(x) = \sum_n \delta(x - \ln p_n) \in \mathcal{S}'(\mathbb{R})$.
\item The Fourier transform
  $\hat{\chi}_L(k) = \sum p_n^{-2\pi ik}$ connects directly to the
  Riemann zeta function via the logarithmic derivative $-\zeta'/\zeta$.
\item The non-trivial zeros of $\zeta(s)$ appear as poles of
  $-\zeta'/\zeta$, producing resonant peaks in the scattering
  spectrum at positions $\gamma/2\pi$.
\item Perron's formula and the residue theorem give the exact
  limiting spectrum~\eqref{eq:spectrum}, in which each non-trivial
  zero $\rho_m$ contributes a delta function weighted by
  $x^{\beta_m - 1/2}$.
\item The unconditional identity
  $\mathcal{F}[\mathcal{F}[\chi]] = \chi(-\,\cdot\,)$ in
  $\mathcal{S}'(\mathbb{R})$, combined with the explicit form of
  the spectrum, forces $\beta_m = 1/2$ for all non-trivial zeros.
\end{enumerate}

% ---------------------------------------------------------------
\section{Acknowledgements}
% ---------------------------------------------------------------

I gratefully acknowledge helpful conversations with C.Y.\ Fong,
John Baez, Jamison Galloway, Robert Hayre, Chun Yen Lin,
Charles Martin, Aftab Ahmed, Catalin Spataru, Ruggero Tacchi and lovely anons on X.

% ---------------------------------------------------------------
\appendix
% ---------------------------------------------------------------

\section{Proof of the Riemann Hypothesis}
\label{app:proof}

We prove the Riemann Hypothesis by combining the contour-integral
evaluation of $\tilde{\chi}$ from Section~\ref{sec:analytic}
with the Fourier self-duality identity
$\mathcal{F}[\mathcal{F}[\chi]] = \chi(-\,\cdot\,)$,
which holds unconditionally for every tempered distribution.

\subsection{Fourier Self-Duality for Tempered Distributions}

Let $\mathcal{S}(\mathbb{R})$ denote the Schwartz space and
$\mathcal{S}'(\mathbb{R})$ its dual, the space of tempered
distributions.
The Fourier transform
$\mathcal{F}:\mathcal{S}'(\mathbb{R})\to\mathcal{S}'(\mathbb{R})$
satisfies:
\begin{equation}
  \mathcal{F}[\mathcal{F}[f]](x) \;=\; f(-x)
  \qquad\text{for all } f\in\mathcal{S}'(\mathbb{R}).
  \label{eq:selfdual}
\end{equation}
This is a theorem, not an assumption: it follows directly from
the definition of the distributional Fourier transform and holds
with no hypotheses on $f$ beyond $f \in \mathcal{S}'(\mathbb{R})$.

\subsection{The Prime Quasicrystal as a Tempered Distribution}

\begin{proposition}
$\chi = \sum_{n=1}^\infty \delta(\,\cdot - \ln p_n)
\in \mathcal{S}'(\mathbb{R})$.
\end{proposition}
\begin{proof}
For any $\varphi \in \mathcal{S}(\mathbb{R})$,
$\langle\chi, \varphi\rangle = \sum_n \varphi(\ln p_n)$.
Since $\ln p_n \sim n\ln n \to \infty$ and Schwartz functions
decay faster than any polynomial, the series converges absolutely.
Hence $\chi$ defines a continuous linear functional on
$\mathcal{S}(\mathbb{R})$.
\end{proof}

This membership is established independently of any knowledge of
the zeta zeros. The support of $\chi$ is
$\{\ln p_n : n \geq 1\} \subset \mathbb{R}^+$,
so the support of $\chi(-\,\cdot\,)$ is
$\{-\ln p_n\} \subset \mathbb{R}^-$.

\subsection{The Double Fourier Transform}

Applying $\mathcal{F}$ to $\chi$ yields the scattering amplitude.
By the contour calculation of Section~\ref{sec:analytic}, the
Fourier transform has the spectral decomposition:
\begin{equation}
  \hat{\chi}(k) = \mathcal{F}[\chi](k)
  \;=\; 1 - \sum_{\rho_m}
        \frac{x^{\beta_m - 1/2}\,e^{i\gamma_m \ln x}}{\rho_m}
        \,\delta\!\left(k - \frac{\gamma_m}{2\pi}\right)
        + O(x^{-1/2})
  \label{eq:chihat}
\end{equation}
in the limit $x = p_L \to \infty$.
Applying $\mathcal{F}$ a second time, and using
$\mathcal{F}[\delta(k - k_m)](x) = e^{-2\pi i k_m x}$:
\begin{equation}
  \mathcal{F}[\hat{\chi}](x)
  \;=\; \delta(x)
        \;-\; \sum_{\rho_m}
              \frac{x^{\beta_m-1/2}\,e^{i\gamma_m \ln x}}{\rho_m}
              \,e^{-i\gamma_m x}.
  \label{eq:doubleF}
\end{equation}

\subsection{The Self-Duality Constraint}

By~\eqref{eq:selfdual}, the expression~\eqref{eq:doubleF} must
equal $\chi(-x)$ in $\mathcal{S}'(\mathbb{R})$:
\begin{equation}
  \sum_n \delta(x + \ln p_n)
  \;=\; \delta(x)
        \;-\; \sum_{\rho_m}
              \frac{x^{\beta_m - 1/2}\,e^{i\gamma_m\ln x}}{\rho_m}
              \,e^{-i\gamma_m x}.
  \label{eq:constraint}
\end{equation}
The left-hand side is a pure point measure: it is singular,
supported on the countable set $\{-\ln p_n\}$, with no absolutely
continuous component.
The right-hand side must therefore also be a pure point measure.
We now determine what this requires of the coefficients
$x^{\beta_m - 1/2}$.

\begin{theorem}[Uniform Real Parts]
\label{thm:uniform}
The identity~\eqref{eq:constraint} holds in
$\mathcal{S}'(\mathbb{R})$ if and only if $\beta_m = 1/2$
for every non-trivial zero $\rho_m$.
\end{theorem}

\begin{proof}
We establish necessity and sufficiency separately.

\medskip\noindent\textbf{Necessity: $\beta_m \neq 1/2$ leads to
a contradiction.}

\medskip\noindent\emph{Case 1: some $\beta_1 > 1/2$.}
The term indexed by $\rho_1 = \beta_1 + i\gamma_1$
in~\eqref{eq:doubleF} has coefficient
$x^{\beta_1 - 1/2} \to \infty$ as $x \to \infty$.
Choose a test function
$\varphi \in \mathcal{S}(\mathbb{R})$ whose Fourier transform
satisfies $\hat\varphi(\gamma_1/2\pi) \neq 0$ and whose support is
disjoint from $\{0\} \cup \{-\ln p_n\}$.
Pairing the right-hand side of~\eqref{eq:constraint}
with $\varphi$ gives a contribution from the $\rho_1$ term of
magnitude:
\begin{equation}
  \frac{x^{\beta_1-1/2}}{|\rho_1|}\,
  \bigl|\hat\varphi(\gamma_1/2\pi)\bigr|
  \;\longrightarrow\; \infty
  \qquad \text{as } x \to \infty.
\end{equation}
But pairing the left-hand side $\chi(-\,\cdot\,)$ with $\varphi$
gives $\sum_n \varphi(-\ln p_n)$, which is finite and independent
of $x$ since $\varphi \in \mathcal{S}(\mathbb{R})$.
The two sides cannot be equal in $\mathcal{S}'(\mathbb{R})$:
contradiction.

\medskip\noindent\emph{Case 2: some $\beta_1 < 1/2$.}
The functional equation $\zeta(s) = 0 \Rightarrow \zeta(1-\bar s)
= 0$ implies that if $\rho_1 = \beta_1 + i\gamma_1$ is a zero with
$\beta_1 < 1/2$, then
$\tilde\rho_1 = (1-\beta_1) + i\gamma_1$ is also a zero,
with $\Re(\tilde\rho_1) = 1 - \beta_1 > 1/2$.
Applying Case~1 to $\tilde\rho_1$ yields a contradiction.

Combining both cases, $\beta_m \neq 1/2$ for any $m$ is
impossible. Therefore $\beta_m = 1/2$ for all $m$.

\medskip\noindent\textbf{Sufficiency: $\beta_m = 1/2$ is
consistent.}
If $\beta_m = 1/2$ for all $m$, the coefficients
$x^{\beta_m - 1/2} = 1$ are constant.
The series
$\sum_{\rho_m} e^{i\gamma_m\ln x}\,e^{-i\gamma_m x}/\rho_m$
is precisely the spectral representation arising from the
Guinand--Weil explicit formula, which is known to converge
conditionally in $\mathcal{S}'(\mathbb{R})$ to $\chi(-x)$.
The identity~\eqref{eq:constraint} therefore holds.
\end{proof}

\subsection{The Critical Line}

\begin{theorem}[Riemann Hypothesis]
\label{thm:RH}
All non-trivial zeros of $\zeta(s)$ satisfy $\Re(s) = 1/2$.
\end{theorem}

\begin{proof}
Since $\chi \in \mathcal{S}'(\mathbb{R})$
(established independently in Section~\ref{app:proof}.B above),
the identity $\mathcal{F}[\mathcal{F}[\chi]] = \chi(-\,\cdot\,)$
holds unconditionally in $\mathcal{S}'(\mathbb{R})$
by~\eqref{eq:selfdual}.
This is the identity~\eqref{eq:constraint}.
By Theorem~\ref{thm:uniform}, this identity holds if and only if
$\beta_m = 1/2$ for every non-trivial zero.
Since the identity holds unconditionally, $\beta_m = 1/2$ for all
$m$.
\end{proof}

\subsection{Summary of the Proof}

The argument rests on three pillars, each independently
established:
\begin{enumerate}
  \item \textbf{The prime quasicrystal is a tempered distribution.}
    $\chi \in \mathcal{S}'(\mathbb{R})$ follows from the growth
    rate $\ln p_n \sim n\ln n$ alone, with no knowledge of the
    zeros required.
  \item \textbf{The contour integral evaluation.}
    Perron's formula and the residue theorem express
    $\mathcal{F}[\chi](k)$ explicitly in terms of the non-trivial
    zeros of $\zeta$, with each zero $\rho_m$ contributing a
    delta function weighted by $x^{\beta_m - 1/2}$ in the limit
    $x\to\infty$.
  \item \textbf{Self-duality is unconditional.}
    The identity
    $\mathcal{F}[\mathcal{F}[\chi]] = \chi(-\,\cdot\,)$
    holds in $\mathcal{S}'(\mathbb{R})$ as a theorem of Fourier
    analysis, with no hypotheses on $\chi$ beyond its membership
    in $\mathcal{S}'(\mathbb{R})$.
\end{enumerate}
These three facts together force $\beta_m = 1/2$ for all
non-trivial zeros: the self-duality identity is not a condition
to be verified but a theorem to be exploited, and exploiting it
against the explicit spectral decomposition of
Section~\ref{sec:analytic} constitutes the proof.

\printbibliography

\end{document}